\chapter{Összegzés} % Conclusion
\label{ch:sum}

Az agilis módszertanban ugyan kisebb létszámú csapatok dolgoznak együtt, mégis rengeteg és szerteágazó feladatokkal találják magukat szemben. A hatékony működés érdekében találták ki ezeket a módszertanokat és azok altípusait (példuál a scrum-ot). 

A gyakorlatban viszont külön eszközökre is szükségük van ezek megvalósításához. A szakdolgozat célja egy olyan alkalmazás elkészítése volt, amely önmagában is alkalmas kisebb scrum-ok fejlesztési munkájának nyomonkövetésére, segítésére. 

Természetesen sokféle mód van még az alkalmazás további fejlődésére, ehhez biztosít stabil alapokat maga a keretrendszer is. Pár ötlet ezek közül röviden az alábbi szekcióban olvasható.

\section{További fejlesztési lehetőségek}

\begin{enumerate}
    \item Egy kereső funkció implemetálása, amely az oldal egész tartalmán végig halad és ezzel megkönnyíti a felhasználóknak az egyes funkciók és objektumok elérését.
    \item Még részletesebb statisztikai oldalak implementálása Sprint specifikusabban (grafikonok, diagrammok, szűrések).
    \item A nézetek (view-k) áthelyezése teljes mértékben a szolgáltatás rétegbe. Vannak applikációk, ahol ez már megvalósításra került (\textit{service.py} modulok) bizonyos mértékben. Ennek célja egy API (Application Programming Interface) kialakítása, ezzel teljesen szinte teljesen függetlenítve a szerver és kliens réteget.
    A Django is rendelkezik egy RESTful kommunikációt megvalósító könyvtárral.
\end{enumerate}




