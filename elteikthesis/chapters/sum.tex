\chapter{Összegzés} % Conclusion
\label{ch:sum}

Az agilis módszertanban ugyan kisebb létszámú csapatok dolgoznak együtt, mégis rengeteg és szerteágazó feladatokkal találják magukat szemben. A hatékony működés érdekében találták ki ezeket a módszertanokat és azok altípusait (például a scrum-ot). 

A gyakorlatban viszont külön eszközökre is szükségük van ezek megvalósításához. A szakdolgozat célja egy olyan alkalmazás elkészítése volt, amely önmagában is alkalmas kisebb scrum-ok fejlesztési munkájának nyomon követésére, segítésére. 

Természetesen sokféle mód van még az alkalmazás további fejlődésére, ehhez biztosít stabil alapokat maga a keretrendszer is. Pár ötlet ezek közül röviden az alábbi felsorolásban olvasható.

\begin{enumerate}
    \item Egy kereső funkció implementálása, amely az oldal egész tartalmán végig halad és ezzel megkönnyíti a felhasználóknak az egyes funkciók és objektumok elérését.
    \item Még részletesebb statisztikai oldalak implementálása Sprint specifikusabban (grafikonok, diagramok, szűrések).
    \item API (Application Programming Interface) készítése. Vannak applikációk, ahol ez már megvalósításra került (\textit{services.py} modulok) bizonyos mértékben. Ezzel szinte teljesen függetlenedne a szerver és kliens réteg.
    A Django is rendelkezik egy RESTful kommunikációt megvalósító könyvtárral (django-rest-framework\footnote{\url{https://www.django-rest-framework.org/}}).
\end{enumerate}




