\chapter{Bevezetés} % Introduction
\label{ch:intro}

A mai világban egyre szélesebb körben alkalmazott az agilis módszertan az informatikai megoldásokkal foglalkozó cégeknél. A jelen dolgozat keretein belül elkészült alkamazás (későbbiekben: ScrumHelper) az agilis módszertan szerint működő fejlesztői csapatok (azaz scrum-ok) segítésére született. Manapság sok, különböző ilyen eszköz elérhető, mind fizetős, mind nyílt-forráskódú változatban is. Ezek a projektek egyhamar hatalamsra nőnek, hogy minél több funkciót lássanak el, ezzel bonyolítva kezelésüket.

 Ezen alkalmazás kisebb létszámú (például startup) cégek számára lehet előnyös elsősorban, de természetsen bármilyen scrum berendezkedésű fejlesztői csapatnak megfelelő. Az alkalmazást könnyen és egyértelműen lehet kezelni (részletesebb bemutatása felhasználók számára a \ref{ch:user}. fejezetben olvasható). 

A ScrumHelper a python programozási nyelv segítségével készült el, azon belül is a Django nyílt forráskódú keretrendszerrel, mely webes alkalmazások fejlesztésére szolgál, kihasználva a python nyelv adta  sokdoldalú lehetőségeket. A teljesség igénye nélkül pár példa ezen előnyök közül: 
\begin{compactitem}
	\item kevés soros és átlátható  kódbázis segíti a gyorsabb fejlesztést
	\item jól dokumentált keretrendszer \footnote{\url{https:/docs.djangoproject.com/en/3.0/}}
	\item modulokra (applikációkra) bontott szoftver segíti az újra felhasználhatóságot
\end{compactitem}

