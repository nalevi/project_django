\chapter{Felhasználói dokumentáció} % User guide
\label{ch:user}

Ezen fejezet a felhasználó részletes tájékoztatására szolgál. Az alfejezetek az alkalmazás szükséges előfeltételeit, telepítési és használati információkat tartalmaznak. A program technikai részletekbe menő dokumentációját a \ref{ch: impl} . fejezet (Fejlesztői dokumentáció) tartalmazza.


\section{Röviden az agilis módszertanról és a  "scrum"-ról} % Briefly about the "scrum" 

\begin{description}
	\item[Mi az agilis módszertan?] A szoftverfejlesztési módszerek egy csoportja, ahol a követelmények és megoldások szoros együttműködésén keresztül fejlődnek az önszerveződő  és 		     	multifonkcionális csapatok. Ez elősegíti az korai szállítást, folytonos továbbfejlesztést és bátorít a változásokra adható gyors és rugalmas válaszokra. \footnote{Részletes leírás 		 	   	megtalálható: \url{https://www.agilealliance.org/agile101/}}
	\item[Mi a "scrum"?] Az agilis módszertanon belül sokféle irányzat van, ezek egyike a scrum. A scrum középpontjában a kis létszámú, önszerveződő agilis csapatok állnak. Itt,
 szemben a többi agilis ágazattal, nincsenek általánosan megszabott ,egész rendszert egybefogó szállítási és frissítési időpontok, hanem az aktuális fejlesztések Sprintekbe rendeződnek és ezek lejárta után történik az átadás. Ez közvetlenebb visszajelzést is biztosít a szállító és a kliens között. A scrum szerkezeti felépítése a következő: van egy scrum master, aki egybe fogja a csapat működését, feldata a scrum "menedzselése" gyakorlatilag. Mellete a csapat tartalmaz természetesen fejlesztőket és tesztelőket. A scrum létszáma nincsen hivatalosan meghatározva, de ajánlatos 8-10 főnél nem nagyobbra nőnie, ugyanis így veszít hatékonyságából. A csapatok minden nap egy meghatározott időpontban tartanak stand up-okat, amelyeken minden tag beszámol feladatairól, haldásáról, így mindenki a csapaton belül képbe kerülhet a Sprint aktuális állásáról.
\end{description}

Pár fontosabb fogalom, amely a későbbiekben külön fejezetben részletezve lesz:
\begin{itemize}
	\item Projekt \ref{Projekt}
	\item Epic \ref{Epic}
	\item User stories \ref{User story}
	\item Task \ref{Task}
	\item Issue \ref{Issue}
	\item Kanban \ref{Kanban}
\end{itemize}

\section{Rendszerkövetelmények} % System requirements

\textbf{Minimum követelmények}: A ScumHelper egy webes alkalmazás. Ez azt jelenti, hogy a felhasználónak csak egy böngészőre van szüksége a számítógépén (például: Google Chrome, Mozilla Firefox, Safari, stb.) és természetesn internet elérésre ahhoz, hogy futtatni tudja a szoftvert. Utóbbi elengedhetetlen, ugyanis csak bejelentkezve lehetséges használni, továbbá internet kapcsolat nélkül a számítógép csak a gyorsírótárában mentett oldalakat tudja megnyitni, de módosításokat nem tudunk végrehajtani az oldalon és váltani sem tudunk az alkalmazáson belül. Nincsen megkötés operációs rendszer tekintetében sem, tehát minden, napjainkban használatos rendszeren (Linux, Windows, Mac OS) egyaránt használható.  

\textbf{Ajánlott követelmények}: A jobb felhasználói élmény érdekében érdemes legalább 1280x720 felbontású kijelzőn használni és a korábban már felsorolt, jelenleg leggyorsabbnak és legbiztonságosabbnak számító böngészővel megnyitni : Chrome, Mozilla firefox, Chromium.

\section{Telepítés}

A felhasználói oldalról nem igényel telepítést. Az eléréshez a szerver elérési URL-jére van szükség, illetve egy felhasználó igénylésére az aktuális rendszergazdától. Érdemes az új felhasználóba való belépés után megváltoztatni jelszavunkat biztonsági okokból.

 \section{Az alkalmazás felépítése}
